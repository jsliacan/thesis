
\newtheorem{theorem}{Theorem}[section]
\newtheorem{proposition}{Proposition}[section]
\newtheorem{corollary}{Corollary}[section]
\newtheorem{lemma}{Lemma}[section]
\theoremstyle{definition}
\newtheorem{definition}{Definition}[section]
\newtheorem{example}{Example}[section]
\newtheorem*{note}{Note}
\newtheorem{remark}{Remark}[section]
\newtheorem{observation}{Observation}[section]

\definecolor{keywords}{RGB}{255,0,90}
\definecolor{comments}{RGB}{0,0,113}
\definecolor{myred}{RGB}{160,0,0}
\definecolor{green}{RGB}{0,150,0}


%% For typesetting code listings                                                
\lstdefinelanguage{Sage}[]{Python}
{morekeywords={False,sage,True},sensitive=true}
\newcommand{\lstsetsage}{\lstset{
  frame=single,
  showtabs=False,
  showspaces=False,
  showstringspaces=False,
  commentstyle={\ttfamily\color{dgraycolor}},
  keywordstyle={\ttfamily\color{dbluecolor}\bfseries},
  stringstyle={\ttfamily\color{dgreencolor}\bfseries},
  language=Sage,
  basicstyle={\fontsize{9pt}{9pt}\ttfamily},
  aboveskip=0.3em,
  belowskip=0.1em,
  numbers=left,
  numberstyle=\tiny
}}

\newcommand{\lstsetbash}{\lstset{
    language=bash,
    basicstyle={\fontsize{9pt}{9pt}\ttfamily},
    deletekeywords={local, for, help},
    keywordstyle=\color{keywords},
    commentstyle=\color{comments},
    stringstyle=\color{black},
    numbers=none,
    stepnumber=1,
    frame=single,
    showstringspaces=false,
}}

\definecolor{dblackcolor}{rgb}{0.0,0.0,0.0}
\definecolor{dbluecolor}{rgb}{0.01,0.02,0.7}
\definecolor{dgreencolor}{rgb}{0.2,0.4,0.0}
\definecolor{dgraycolor}{rgb}{0.30,0.3,0.30}
\newcommand{\dblue}{\color{dbluecolor}\bf}
\newcommand{\dred}{\color{dredcolor}\bf}
\newcommand{\dblack}{\color{dblackcolor}\bf}


\newcommand{\shellcmd}[1]{\texttt{\footnotesize\$ #1}}
\renewcommand{\vert}[1]{\ensuremath{\llbracket #1 \rrbracket}} % Razborov's averaging operator
\newcommand{\Av}{\ensuremath{\mathrm{Av}}}
\newcommand{\Si}{\ensuremath{\mathrm{Si}}}
\newcommand{\seq}[1]{\text{\textsc{Seq}}[#1]}
\newcommand{\D}[1]{\ensuremath{\mathrm{\textbf{D}}}}
\renewcommand{\P}{\ensuremath{\mathcal{P}}}
\newcommand{\T}{\ensuremath{\mathcal{T}}}
\newcommand{\V}{\ensuremath{\mathcal{V}}}
\newcommand{\B}{\ensuremath{\mathcal{B}}}
\newcommand{\Z}{\ensuremath{\mathcal{Z}}}
\newcommand{\Zstar}{\ensuremath{\mathcal{Z}^*}}
\newcommand{\M}{\ensuremath{\mathcal{M}}}
\newcommand{\Mstar}{\ensuremath{\mathcal{M}^*}}
\newcommand{\barMstar}{\ensuremath{\overline{\mathcal{M}}^*}}
\newcommand{\EE}{\ensuremath{\mathcal{E}}}
\newcommand{\C}{\ensuremath{\mathcal{C}}}
\newcommand{\Cstar}{\ensuremath{\mathcal{C}^*}}
\newcommand{\HH}{\ensuremath{\mathcal{H}}}
\newcommand{\HHstar}{\ensuremath{\mathcal{H}^*}}
\newcommand{\hatC}{\ensuremath{\widehat{\mathcal{C}}}}
\newcommand{\hatCstar}{\ensuremath{\widehat{\mathcal{C}}^*}}
\newcommand{\barC}{\ensuremath{\overline{\mathcal{C}}}}
\newcommand{\barCstar}{\ensuremath{\overline{\mathcal{C}}^*}}
\newcommand{\Q}{\ensuremath{\mathcal{Q}}}
\newcommand{\F}{\ensuremath{\mathcal{F}}}
\renewcommand{\S}{\ensuremath{\mathcal{S}}}
\newcommand{\Sstar}{\ensuremath{\mathcal{S}^*}}
\renewcommand{\SS}{\ensuremath{\mathbb{S}}}
\newcommand{\rS}{\ensuremath{\mathscr{S}}}
\newcommand{\DD}{\ensuremath{\mathcal{D}}}
\newcommand{\DDstar}{\ensuremath{\mathcal{D}^*}}
\newcommand{\RR}{\ensuremath{\mathbb{R}}}

% Serif font
\newcommand{\fL}{\ensuremath{\mathsf{L}}}
\newcommand{\fR}{\ensuremath{\mathsf{R}}}
\newcommand{\fD}{\ensuremath{\mathsf{D}}}
\newcommand{\fU}{\ensuremath{\mathsf{U}}}
\newcommand{\fC}{\ensuremath{\mathsf{C}}}
\newcommand{\fB}{\ensuremath{\mathsf{B}}}
\newcommand{\fS}{\ensuremath{\mathsf{S}}}
\newcommand{\fM}{\ensuremath{\mathsf{M}}}
\newcommand{\fE}{\ensuremath{\mathsf{E}}}
\newcommand{\fX}{\ensuremath{\mathsf{X}}}

% bold Serif font
\newcommand{\bD}{\ensuremath{\textbf{\textsf{D}}}}
\newcommand{\bU}{\ensuremath{\textbf{\textsf{U}}}}
\newcommand{\bC}{\ensuremath{\textbf{\textsf{C}}}}
\newcommand{\bE}{\ensuremath{\textbf{\textsf{E}}}}
\newcommand{\bbU}{\ensuremath{\mathbb{U}}}
\newcommand{\bM}{\ensuremath{\textbf{\textsf{M}}}}
\renewcommand{\bm}{\ensuremath{\mathbf{m}}}
\newcommand{\bc}{\ensuremath{\mathbf{c}}}
\newcommand{\be}{\ensuremath{\mathbf{e}}}

\newcommand{\distav}[2]{\ensuremath{\mathbb{A}_{#1}(#2)}}
\newcommand{\N}{\ensuremath{\mathbb{N}}}
\newcommand{\rhs}{\ensuremath{\mathrm{RHS}}}
\newcommand{\lhs}{\ensuremath{\mathrm{LHS}}}
\newcommand{\argmax}[1]{\ensuremath{\mathrm{argmax}(#1)}}
\newcommand{\argmin}[1]{\ensuremath{\mathrm{argmin}(#1)}}
\newcommand{\Prob}[1]{\ensuremath{\mathbf{Pr}\left(#1\right)}}
\newcommand{\E}[1]{\ensuremath{\mathbf{E}\left(#1\right)}}
\newcommand{\Exp}[2]{\ensuremath{\mathbf{E}_{#1}\left(#2\right)}}
\newcommand{\1}[1]{\ensuremath{\mathbbm{1}_{#1}}}
\newcommand{\2}[2]{\ensuremath{\mathbbm{1}_{#1}\left(#2\right)}}
\newcommand{\Space}{\ensuremath{\mathcal{S}}}
\newcommand{\powerset}[1]{\ensuremath{\mathcal{P}\left(#1\right)}}
\newcommand{\mg}[1]{\ensuremath{\mathrm{mg}\left(#1\right)}}
\newcommand{\gr}[1]{\ensuremath{\mathrm{gr}\left(#1\right)}}
\newcommand{\ugr}[1]{\ensuremath{\overline{\mathrm{gr}}\left(#1\right)}}
\newcommand{\ex}{\ensuremath{\mathrm{ex}}}
\newcommand{\im}{\ensuremath{\mathrm{im}}}
\newcommand{\bigominus}{\ensuremath{\mathlarger{\mathlarger{\mathlarger{\ominus}}}}}
\newcommand{\x}{\ensuremath{\mathbf{x}}}
\newcommand{\Grid}{\ensuremath{\mathrm{Grid}}}


%%%%%%%%%%%%%%%%%%%%%%%%%%%%%%%%%%%%%%%%%
% DYCK PATHS
%%%%%%%%%%%%%%%%%%%%%%%%%%%%%%%%%%%%%%%%%


% drawing Dyck paths:
\newcommand\dyck[4]{
  % start point, size, Dyck word (size x 2 booleans)
  \fill[white]  (#1) rectangle +(#2,#2);
  \draw[help lines] (#1) grid +(#2,#2);
  %\draw[dashed] (#1) -- +(#2,#2);
  \coordinate (prev) at (#1);
  \foreach \dir in {#4}{
    \ifnum\dir=0
    \coordinate (dep) at (1,0);
    \else
    \coordinate (dep) at (0,1);
    \fi
    \draw[line width=2pt, color=#3, cap=round] (prev) -- ++(dep) coordinate (prev);
  };
}


% drawing Dyck paths without grid
\newcommand\pdyck[3]{
  % start point, size, Dyck word (size x 2 booleans)
  \coordinate (prev) at (#1);
  \foreach \dir in {#3}{
    \ifnum\dir=0
    \coordinate (dep) at (1,0);
    \else
    \coordinate (dep) at (0,1);
    \fi
    \draw[line width=2pt, color=#2, cap=round, style=dotted] (prev) -- ++(dep) coordinate (prev);
  };
}


%%%%%%%%%%%%%%%%%%%%%%%%%%%%%%%%%%%%%%%%%%
% CONSTRUCTIONS OF PERMUTONS
%%%%%%%%%%%%%%%%%%%%%%%%%%%%%%%%%%%%%%%%%%
\newcommand{\acbmax}{
  \begin{tikzpicture}[scale=0.5]
    \draw (0,1.7)--(1.3,3);
    \draw (1.3,1)--(2,1.7);
    \draw (2,0.7)--(2.3,1);
    \draw (2.3,0.5)--(2.5,0.7);
    \draw (2.5,0.4)--(2.6,0.5);
    \draw (2.65,0.3)--(2.7,0.35);
  \end{tikzpicture}}


\newcommand{\acdbmax}{
  \begin{tikzpicture}[scale=0.4]

    %\node at (-8,0) {$\Gamma\ = $};
    \draw (3,6)--(5,8);
    \draw (5,4.7)--(6.3,6);
    \draw (6.3,4)--(7,4.7);
    \draw (7,3.7)--(7.3,4);
    \draw (7.3,3.5)--(7.5,3.7);
    \draw (7.5,3.4)--(7.6,3.5);
    \draw (7.65,3.3)--(7.7,3.35);

    \draw (0,1.7)--(1.3,3);
    \draw (1.3,1)--(2,1.7);
    \draw (2,0.7)--(2.3,1);
    \draw (2.3,0.5)--(2.5,0.7);
    \draw (2.5,0.4)--(2.6,0.5);
    \draw (2.65,0.3)--(2.7,0.35);

    \draw (-1.7,-0.7)--(-1,0);
    \draw (-1,-1)--(-0.7,-0.7);
    \draw (-0.7,-1.2)--(-0.5,-1);
    \draw (-0.5,-1.3)--(-0.4,-1.2);
    \draw (-0.35,-1.4)--(-0.3,-1.35);

    \draw (-2.7,-2.3)--(-2.4,-2);
    \draw (-2.4,-2.5)--(-2.2,-2.3);
    \draw (-2.2,-2.6)--(-2.1,-2.5);
    \draw (-2.05,-2.7)--(-2,-2.65);

    \draw[thick] (-3.05,-3.05)--(-3,-3);
    \draw[thick] (-3.25,-3.25)--(-3.2,-3.2);
    \draw[thick] (-3.45,-3.45)--(-3.4,-3.4);


    \draw (14,2)--(16,2);
    \draw (14,1.5)--(16,1.5);

    \draw (25,6)--(27,8)--(29.7,3.35)--(25,6);
    \draw (20,-2.35)--(25,-2.35)--(25,3.35)--(20,3.35)--(20,-2.35);

  \end{tikzpicture}}






\newcommand{\Amax}{
  \begin{tikzpicture}[baseline=1ex, scale=0.15]
    \draw (0,1.7)--(1.3,3);
    \draw (1.3,1)--(2,1.7);
    \draw (2,0.7)--(2.3,1);
    \draw (2.3,0.5)--(2.5,0.7);
    \draw (2.5,0.4)--(2.6,0.5);
    \draw (2.65,0.3)--(2.7,0.35);

    \draw (3,5)--(5,3);
  \end{tikzpicture}}

\newcommand{\AAmax}{
  \begin{tikzpicture}[baseline=1ex, scale=0.15]
    \draw (0,1.7)--(1.3,3);
    \draw (1.3,1)--(2,1.7);
    \draw (2,0.7)--(2.3,1);
    \draw (2.3,0.5)--(2.5,0.7);
    \draw (2.5,0.4)--(2.6,0.5);
    \draw (2.65,0.3)--(2.7,0.35);

    \draw (2.5,5)--(5,2.5);
  \end{tikzpicture}}




\newcommand{\Bmax}{
  \begin{tikzpicture}[baseline=1ex, scale=0.15]
    \draw (3,3)--(5,5);
    \draw (5,1)--(7,3);

    \draw (1,0)--(2,1);
    \draw (2,-1)--(3,0);
    
    \draw (0,-1.5)--(0.5,-1);
    \draw (0.5,-2)--(1,-1.5);

    \draw (-0.4,-2.2)--(-0.2,-2);
    \draw (-0.2,-2.4)--(0,-2.2);

  \end{tikzpicture}}



\newcommand{\Cmax}{
  \begin{tikzpicture}[baseline=1ex, scale=0.15]

    \draw (3,5)--(5,3);
    \draw (2,3)--(3,2);
    \draw (1,2)--(2,1);
    \draw (-1,1)--(1,-1);
  \end{tikzpicture}}




\newcommand{\Dmax}{
  \begin{tikzpicture}[baseline=1ex, scale=0.15]
    \draw (0,1.7)--(1.3,3);
    \draw (1.3,1)--(2,1.7);
    \draw (2,0.7)--(2.3,1);
    \draw (2.3,0.5)--(2.5,0.7);
    \draw (2.5,0.4)--(2.6,0.5);
    \draw (2.65,0.3)--(2.7,0.35);

    \draw (3,4.7)--(4.3,6);
    \draw (4.3,4)--(5,4.7);
    \draw (5,3.7)--(5.3,4);
    \draw (5.3,3.5)--(5.5,3.7);
    \draw (5.5,3.4)--(5.6,3.5);
    \draw (5.65,3.3)--(5.7,3.35);
  \end{tikzpicture}}

\newcommand{\Dmaxr}{
  \begin{tikzpicture}[baseline=1ex, scale=0.15]
    \draw (0,1.7)--(1.3,3);
    \draw (1.3,1)--(2,1.7);
    \draw (2,0.7)--(2.3,1);
    \draw (2.3,0.5)--(2.5,0.7);
    \draw (2.5,0.4)--(2.6,0.5);
    \draw (2.65,0.3)--(2.7,0.35);

    \draw (4.5,3)--(5.8,4.3);
    \draw (3.8,4.3)--(4.5,5);
    \draw (3.5,5)--(3.8,5.3);
    \draw (3.3,5.3)--(3.5,5.5);
    \draw (3.2, 5.5)--(3.3, 5.6);
    \draw (3.15, 5.6)--(3.2, 5.65);
  \end{tikzpicture}}


\newcommand{\Emax}{
  \begin{tikzpicture}[baseline=1ex, scale=0.15]

    \draw (0,2)--(2,0);
    \draw (2,4)--(4,6);
    \draw (4,2)--(6,4);
  \end{tikzpicture}}



% 1342 7term
\newcommand{\acdbmaxapprox}{
  \begin{tikzpicture}[baseline=1ex, scale=0.5]
    \draw (0,1.7)--(1.3,3);
    \draw (1.3,1)--(2,1.7);
    \draw (2,0.7)--(2.3,1);
    \draw (2.3,0.5)--(2.5,0.7);
    \draw (2.5,0.4)--(2.6,0.5);
    \draw (2.65,0.3)--(2.7,0.35);
  \end{tikzpicture}}


%%%%%%%%%%%%%%%%%%%%%%%%%%%%%%%%%%%%%%%%%%%%
% SMALL PERMUTATION PICTOGRAMS
%%%%%%%%%%%%%%%%%%%%%%%%%%%%%%%%%%%%%%%%%%%%

\newcommand{\dicycle}{
  \begin{tikzpicture}[baseline=-0.3ex,scale=0.2]
    \tikzstyle{vertex}=[circle,fill=black, minimum size=1pt,inner sep=1pt]
    \node[vertex] (v1) at (0,0){};
    \node[vertex] (v2) at (2,0){};
    \node[vertex] (v3) at (1,1.4){};
    \draw[->](v1)--(v2);
    \draw[->](v2)--(v3);
    \draw[->](v3)--(v1);
  \end{tikzpicture}
}

\newcommand{\twochain}{
  \begin{tikzpicture}[baseline=-0.3ex,scale=0.2]
    \tikzstyle{vertex}=[circle,fill=black, minimum size=1pt,inner sep=1pt]
    \node[vertex] (v1) at (0,0){};
    \node[vertex] (v2) at (2,0){};
    \node[vertex] (v3) at (1,1.4){};
    \draw[->](v1)--(v2);
    \draw[->](v2)--(v3);
  \end{tikzpicture}
}

\newcommand{\orcocherry}{
  \begin{tikzpicture}[baseline=-0.3ex,scale=0.2]
    \tikzstyle{vertex}=[circle,fill=black, minimum size=1pt,inner sep=1pt]
    \node[vertex] (v1) at (0,0){};
    \node[vertex] (v2) at (2,0){};
    \node[vertex] (v3) at (1,1.4){};
    \draw[->](v1)--(v2);
  \end{tikzpicture}
}

\newcommand{\outstar}{
  \begin{tikzpicture}[baseline=-0.3ex,scale=0.2]
    \tikzstyle{vertex}=[circle,fill=black, minimum size=1pt,inner sep=1pt]
    \node[vertex] (v1) at (0,0){};
    \node[vertex] (v2) at (2,0){};
    \node[vertex] (v3) at (1,1.4){};
    \draw[->](v3)--(v1);
    \draw[->](v3)--(v2);
  \end{tikzpicture}
}


\newcommand{\digraphacbd}{
  \begin{tikzpicture}[baseline=-0.3ex,scale=0.2]
    \tikzstyle{vertex}=[circle,fill=black, minimum size=1pt,inner sep=1pt]
    \node[vertex] (v1) at (0,0){};
    \node[vertex] (v2) at (2,0){};
    \node[vertex] (v3) at (1,1.4){};
    \node[vertex] (v4) at (3, 1.4){};
    \draw[->](v1)--(v2);
    \draw[->](v1)--(v3);
    \draw[->](v1)--(v4);
    \draw[->](v2)--(v4);
    \draw[->](v3)--(v4);
  \end{tikzpicture}
}

%\input struct.tex
\newcommand{\gridbadc}{
  \begin{tikzpicture}[baseline=0.5ex,scale=0.15]
  \tikzstyle{vertex}=[circle,draw=black, fill=black, minimum size=2pt,inner sep=1pt]

  \draw[gray, very thin] (0,0) grid (9,9);
  \fill[gray] (1,0) rectangle (2,9);
  \fill[gray] (3,0) rectangle (4,9);
  \fill[gray] (5,0) rectangle (6,9);
  \fill[gray] (7,0) rectangle (8,9);

  \fill[gray] (0,1) rectangle (9,2);
  \fill[gray] (0,3) rectangle (9,4);
  \fill[gray] (0,5) rectangle (9,6);
  \fill[gray] (0,7) rectangle (9,8);
  
  \node[vertex] (v1) at (1.5, 3.5){};
  \node[vertex] (v2) at (3.5,1.5){};
  \node[vertex] (v3) at (5.5,7.5){};
  \node[vertex] (v4) at (7.5,5.5){};
  \draw (v1) (v2) (v3) (v4);
  \end{tikzpicture}}



\newcommand{\gridabdc}{
  \begin{tikzpicture}[baseline=0.5ex,scale=0.15]
  \tikzstyle{vertex}=[circle,draw=black, fill=black, minimum size=2pt,inner sep=1pt]

  \draw[gray, very thin] (0,0) grid (9,9);
  \fill[gray] (1,0) rectangle (2,9);
  \fill[gray] (3,0) rectangle (4,9);
  \fill[gray] (5,0) rectangle (6,9);
  \fill[gray] (7,0) rectangle (8,9);

  \fill[gray] (0,1) rectangle (9,2);
  \fill[gray] (0,3) rectangle (9,4);
  \fill[gray] (0,5) rectangle (9,6);
  \fill[gray] (0,7) rectangle (9,8);
  
  \node[vertex] (v1) at (1.5, 1.5){};
  \node[vertex] (v2) at (3.5,3.5){};
  \node[vertex] (v3) at (5.5,7.5){};
  \node[vertex] (v4) at (7.5,5.5){};
  \draw (v1) (v2) (v3) (v4);
  \end{tikzpicture}}



\newcommand{\gridbacd}{
  \begin{tikzpicture}[baseline=0.5ex,scale=0.15]
  \tikzstyle{vertex}=[circle,draw=black, fill=black, minimum size=2pt,inner sep=1pt]

  \draw[gray, very thin] (0,0) grid (9,9);
  \fill[gray] (1,0) rectangle (2,9);
  \fill[gray] (3,0) rectangle (4,9);
  \fill[gray] (5,0) rectangle (6,9);
  \fill[gray] (7,0) rectangle (8,9);

  \fill[gray] (0,1) rectangle (9,2);
  \fill[gray] (0,3) rectangle (9,4);
  \fill[gray] (0,5) rectangle (9,6);
  \fill[gray] (0,7) rectangle (9,8);
  
  \node[vertex] (v1) at (1.5, 3.5){};
  \node[vertex] (v2) at (3.5,1.5){};
  \node[vertex] (v3) at (5.5,5.5){};
  \node[vertex] (v4) at (7.5,7.5){};
  \draw (v1) (v2) (v3) (v4);
  \end{tikzpicture}}




%%%%%%%%%%%%%%%%%%%%%%%%%%%%%%%%%%%%%%%%%%%%%%%%%%%%%%%%%%%%%%%%%%%%%%%%%%%%%%%%%%%
% 1-POINT PERMTUATIONS
%%%%%%%%%%%%%%%%%%%%%%%%%%%%%%%%%%%%%%%%%%%%%%%%%%%%%%%%%%%%%%%%%%%%%%%%%%%%%%%%%%%



\newcommand{\atau}{
  \begin{tikzpicture}[baseline=0.5ex,scale=0.15]
  \tikzstyle{vertex}=[circle,draw=black,fill=white, minimum size=2pt,inner sep=1pt]
  \node[vertex] (v1) at (0.5, 0.5){};
  \draw (v1);
  \end{tikzpicture}}

\renewcommand{\a}{
  \begin{tikzpicture}[baseline=0.5ex,scale=0.15]
  \tikzstyle{vertex}=[circle,fill=black, minimum size=2pt,inner sep=1pt]
  \node[vertex] (v1) at (0.5, 0.5){};
  \draw (v1);
  \end{tikzpicture}}


%%%%%%%%%%%%%%%%%%%%%%%%%%%%%%%%%%%%%%%%%%%%%%%%%%%%%%%%%%%%%%%%%%%%%%%%%%%%%%%%%%%
% 2-POINT PERMTUATIONS
%%%%%%%%%%%%%%%%%%%%%%%%%%%%%%%%%%%%%%%%%%%%%%%%%%%%%%%%%%%%%%%%%%%%%%%%%%%%%%%%%%%

\newcommand{\abtau}{
  \begin{tikzpicture}[baseline=0.5ex,scale=0.15]
  \tikzstyle{vertex}=[circle,draw=black, fill=black, minimum size=2pt,inner sep=1pt]
  \node[vertex] (v1) at (0.5, 0.5){};
  \tikzstyle{vertex}=[circle,draw=black, fill=white, minimum size=2pt,inner sep=1pt]
  \node[vertex] (v2) at (1.5,1.5){};
  %\draw[gray, very thin] (0,0) grid (2,2);
  \draw (v1) (v2);
  \end{tikzpicture}}

\newcommand{\tauab}{
  \begin{tikzpicture}[baseline=0.5ex,scale=0.15]
  \tikzstyle{vertex}=[circle,draw=black, fill=white, minimum size=2pt,inner sep=1pt]
  \node[vertex] (v1) at (0.5, 0.5){};
  \tikzstyle{vertex}=[circle,draw=black, fill=black, minimum size=2pt,inner sep=1pt]
  \node[vertex] (v2) at (1.5,1.5){};
  %\draw[gray, very thin] (0,0) grid (2,2);
  \draw (v1) (v2);
  \end{tikzpicture}}

\newcommand{\tauba}{
  \begin{tikzpicture}[baseline=0.5ex,scale=0.15]
  \tikzstyle{vertex}=[circle, draw=black, fill=white, minimum size=2pt,inner sep=1pt]
  \node[vertex] (v1) at (0.5, 1.5){};
  \tikzstyle{vertex}=[circle, draw=black, fill=black, minimum size=2pt,inner sep=1pt]
  \node[vertex] (v2) at (1.5, 0.5){};
  %\draw[gray, very thin] (0,0) grid (2,2);
  \draw (v1) (v2);
  \end{tikzpicture}}

\newcommand{\batau}{
  \begin{tikzpicture}[baseline=0.5ex,scale=0.15]
  \tikzstyle{vertex}=[circle, draw=black, fill=black, minimum size=2pt,inner sep=1pt]
  \node[vertex] (v1) at (0.5, 1.5){};
  \tikzstyle{vertex}=[circle, draw=black, fill=white, minimum size=2pt,inner sep=1pt]
  \node[vertex] (v2) at (1.5, 0.5){};
  %\draw[gray, very thin] (0,0) grid (2,2);
  \draw (v1) (v2);
  \end{tikzpicture}}


%================ PLAIN ================

\newcommand{\ab}{
  \begin{tikzpicture}[baseline=0.5ex,scale=0.15]
  \tikzstyle{vertex}=[circle,fill=black, minimum size=2pt,inner sep=1pt]
  \node[vertex] (v1) at (0.5, 0.5){};
  \tikzstyle{vertex}=[circle,fill=black, minimum size=2pt,inner sep=1pt]
  \node[vertex] (v2) at (1.5,1.5){};
  %\draw[gray, very thin] (0,0) grid (2,2);
  \draw (v1) (v2);
  \end{tikzpicture}}

\newcommand{\ba}{
  \begin{tikzpicture}[baseline=0.5ex,scale=0.15]
  \tikzstyle{vertex}=[circle, fill=black, minimum size=2pt,inner sep=1pt]
  \node[vertex] (v1) at (1.5, 1.5){};
  \tikzstyle{vertex}=[circle, fill=black, minimum size=2pt,inner sep=1pt]
  \node[vertex] (v2) at (0.5, 0.5){};
  %\draw[gray, very thin] (0,0) grid (2,2);
  \draw (v1) (v2);
  \end{tikzpicture}}



%%%%%%%%%%%%%%%%%%%%%%%%%%%%%%%%%%%%%%%%%%%%%%%%%%%%%%%%%%%%%%%%%%%%%%%%%%%%%%%%%%%
% 3-POINT PERMTUATIONS
%%%%%%%%%%%%%%%%%%%%%%%%%%%%%%%%%%%%%%%%%%%%%%%%%%%%%%%%%%%%%%%%%%%%%%%%%%%%%%%%%%%

\newcommand{\tauabc}{
  \begin{tikzpicture}[baseline=0.5ex,scale=0.15]
  \tikzstyle{vertex}=[circle,draw=black,fill=white, minimum size=2pt,inner sep=1pt]
  \node[vertex] (v1) at (0.5, 0.5){};
  \tikzstyle{vertex}=[circle,fill=black, minimum size=2pt,inner sep=1pt]
  \node[vertex] (v2) at (1.5,1.5){};
  \node[vertex] (v3) at (2.5, 2.5){};
  \draw (v1) (v2) (v3);
  \end{tikzpicture}}

\newcommand{\ataubc}{
  \begin{tikzpicture}[baseline=0.5ex,scale=0.15]
  \tikzstyle{vertex}=[circle,draw=black,fill=white, minimum size=2pt,inner sep=1pt]
  \node[vertex] (v1) at (1.5, 1.5){};
  \tikzstyle{vertex}=[circle,fill=black, minimum size=2pt,inner sep=1pt]
  \node[vertex] (v2) at (0.5,0.5){};
  \node[vertex] (v3) at (2.5, 2.5){};
  \draw (v1) (v2) (v3);
  \end{tikzpicture}}

\newcommand{\acbtau}{
  \begin{tikzpicture}[baseline=0.5ex,scale=0.15]
  \tikzstyle{vertex}=[circle,draw=black,fill=white, minimum size=2pt,inner sep=1pt]
  \node[vertex] (v1) at (0.5, 0.5){};
  \tikzstyle{vertex}=[circle,fill=black, minimum size=2pt,inner sep=1pt]
  \node[vertex] (v2) at (1.5,2.5){};
  \node[vertex] (v3) at (2.5, 1.5){};
  \draw (v1) (v2) (v3);
  \end{tikzpicture}}


\newcommand{\abc}{
  \begin{tikzpicture}[baseline=0.5ex,scale=0.15]
  \tikzstyle{vertex}=[circle,fill=black, minimum size=2pt,inner sep=1pt]
  \node[vertex] (v1) at (0.5, 0.5){};
  \node[vertex] (v2) at (1.5,1.5){};
  \node[vertex] (v3) at (2.5, 2.5){};
  \draw (v1) (v2) (v3);
  \end{tikzpicture}}


\newcommand{\acb}{
  \begin{tikzpicture}[baseline=0.5ex,scale=0.15]
  \tikzstyle{vertex}=[circle,fill=black, minimum size=2pt,inner sep=1pt]
  \node[vertex] (v1) at (0.5, 0.5){};
  \node[vertex] (v2) at (1.5,2.5){};
  \node[vertex] (v3) at (2.5, 1.5){};
  \draw (v1) (v2) (v3);
  \end{tikzpicture}}


\newcommand{\bac}{
  \begin{tikzpicture}[baseline=0.5ex,scale=0.15]
  \tikzstyle{vertex}=[circle,fill=black, minimum size=2pt,inner sep=1pt]
  \node[vertex] (v2) at (0.5,1.5){};
  \node[vertex] (v1) at (1.5, 0.5){};
  \node[vertex] (v3) at (2.5, 2.5){};
  \draw (v1) (v2) (v3);
  \end{tikzpicture}}

\newcommand{\bca}{
  \begin{tikzpicture}[baseline=0.5ex,scale=0.15]
  \tikzstyle{vertex}=[circle,fill=black, minimum size=2pt,inner sep=1pt]
  \node[vertex] (v1) at (0.5, 1.5){};
  \node[vertex] (v2) at (1.5,2.5){};
  \node[vertex] (v3) at (2.5, 0.5){};
  \draw (v1) (v2) (v3);
  \end{tikzpicture}}


\newcommand{\cba}{
  \begin{tikzpicture}[baseline=0.5ex,scale=0.15]
  \tikzstyle{vertex}=[circle,fill=black, minimum size=2pt,inner sep=1pt]
  \node[vertex] (v3) at (0.5, 2.5){};
  \node[vertex] (v2) at (1.5,1.5){};
  \node[vertex] (v1) at (2.5, 0.5){};
  \draw (v1) (v2) (v3);
  \end{tikzpicture}}

\newcommand{\cab}{
  \begin{tikzpicture}[baseline=0.5ex,scale=0.15]
  \tikzstyle{vertex}=[circle,fill=black, minimum size=2pt,inner sep=1pt]
  \node[vertex] (v1) at (0.5, 2.5){};
  \node[vertex] (v2) at (1.5,0.5){};
  \node[vertex] (v3) at (2.5, 1.5){};
  \draw (v1) (v2) (v3);
  \end{tikzpicture}}

%%%%%%%%%%%%%%%%%%%%%%%%%%%%%%%%%%%%%%%%%%%%%%%%%%%%%%%%%%%%%%%%%%%%%%%%%%%%%%%%%%%
% 4-POINT PERMTUATIONS
%%%%%%%%%%%%%%%%%%%%%%%%%%%%%%%%%%%%%%%%%%%%%%%%%%%%%%%%%%%%%%%%%%%%%%%%%%%%%%%%%%%

\newcommand{\abcd}{
  \begin{tikzpicture}[baseline=0.6ex,scale=0.1]
  \tikzstyle{vertex}=[circle,fill=black, minimum size=2pt,inner sep=1pt]
  \node[vertex] (v1) at (0.5, 0.5){};
  \node[vertex] (v2) at (1.5, 1.5){};
  \node[vertex] (v3) at (2.5, 2.5){};
  \node[vertex] (v4) at (3.5, 3.5){};
  \draw (v1) (v2) (v3) (v4);
  \end{tikzpicture}}

\newcommand{\bdca}{
  \begin{tikzpicture}[baseline=0.6ex,scale=0.1]
  \tikzstyle{vertex}=[circle,fill=black, minimum size=2pt,inner sep=1pt]
  \node[vertex] (v1) at (0.5, 1.5){};
  \node[vertex] (v2) at (1.5, 3.5){};
  \node[vertex] (v3) at (2.5, 2.5){};
  \node[vertex] (v4) at (3.5, 0.5){};
  \draw (v1) (v2) (v3) (v4);
  \end{tikzpicture}}

\newcommand{\acdb}{
  \begin{tikzpicture}[baseline=0.6ex,scale=0.1]
  \tikzstyle{vertex}=[circle,fill=black, minimum size=2pt,inner sep=1pt]
  \node[vertex] (v1) at (0.5, 0.5){};
  \node[vertex] (v2) at (1.5, 2.5){};
  \node[vertex] (v3) at (2.5, 3.5){};
  \node[vertex] (v4) at (3.5, 1.5){};
  \draw (v1) (v2) (v3) (v4);
  \end{tikzpicture}}

%%%%%%%%%%%%%%%%%%%%%%%%%%%%%%%%%%%%%%%%%%%%%%%%%%%%%%%%%%%%%%%%%%%%%%%%%%%%%%%%%%%
% 5-POINT PERMTUATIONS
%%%%%%%%%%%%%%%%%%%%%%%%%%%%%%%%%%%%%%%%%%%%%%%%%%%%%%%%%%%%%%%%%%%%%%%%%%%%%%%%%%%

\newcommand{\bcaoba}{
  \begin{tikzpicture}[baseline=0.6ex,scale=0.1]
  \tikzstyle{vertex}=[circle,fill=black, minimum size=2pt,inner sep=1pt]
  \node[vertex] (v1) at (0.5, 1.5){};
  \node[vertex] (v2) at (1.5, 2.5){};
  \node[vertex] (v3) at (2.5, 0.5){};
  \node[vertex] (v4) at (3.5, 4.5){};
  \node[vertex] (v5) at (4.5, 3.5){};
  \draw[gray, very thin] (0,0) grid (5,5);
  \draw (v1) (v2) (v3) (v4) (v5);
  \end{tikzpicture}}


\newcommand{\baoaoba}{
  \begin{tikzpicture}[baseline=0.6ex,scale=0.1]
  \tikzstyle{vertex}=[circle,fill=black, minimum size=2pt,inner sep=1pt]
  \node[vertex] (v1) at (0.5, 1.5){};
  \node[vertex] (v2) at (1.5, 0.5){};
  \node[vertex] (v3) at (2.5, 2.5){};
  \node[vertex] (v4) at (3.5, 4.5){};
  \node[vertex] (v5) at (4.5, 3.5){};
  \draw[gray, very thin] (0,0) grid (5,5);
  \draw (v1) (v2) (v3) (v4) (v5);
  \end{tikzpicture}}


\newcommand{\aobamba}{
  \begin{tikzpicture}[baseline=0.6ex,scale=0.1]
  \tikzstyle{vertex}=[circle,fill=black, minimum size=2pt,inner sep=1pt]
  \node[vertex] (v1) at (0.5, 0.5){};
  \node[vertex] (v2) at (1.5, 3.5){};
  \node[vertex] (v3) at (2.5, 4.5){};
  \node[vertex] (v4) at (3.5, 1.5){};
  \node[vertex] (v5) at (4.5, 2.5){};
  \draw[gray, very thin] (0,0) grid (5,5);
  \draw (v1) (v2) (v3) (v4) (v5);
  \end{tikzpicture}}



%%%%%%%%%%%%%%%%%%%%%%%%%%%%%%%%%%%%%%%%%%%%%%%%%%%%%%%%%%%%%%%%%%%%%%%%%%%%%%%%%%%
% 6-POINT PERMTUATIONS
%%%%%%%%%%%%%%%%%%%%%%%%%%%%%%%%%%%%%%%%%%%%%%%%%%%%%%%%%%%%%%%%%%%%%%%%%%%%%%%%%%%


\newcommand{\bcaocba}{
  \begin{tikzpicture}[baseline=0.6ex,scale=0.1]
  \tikzstyle{vertex}=[circle,fill=black, minimum size=2pt,inner sep=1pt]
  \node[vertex] (v1) at (0.5, 1.5){};
  \node[vertex] (v2) at (1.5, 2.5){};
  \node[vertex] (v3) at (2.5, 0.5){};
  \node[vertex] (v4) at (3.5, 5.5){};
  \node[vertex] (v5) at (4.5, 4.5){};
  \node[vertex] (v6) at (5.5, 3.5){};
  \draw[gray, very thin] (0,0) grid (6,6);
  \draw (v1) (v2) (v3) (v4) (v5) (v6);
  \end{tikzpicture}}


\newcommand{\bcaobca}{
  \begin{tikzpicture}[baseline=0.6ex,scale=0.1]
  \tikzstyle{vertex}=[circle,fill=black, minimum size=2pt,inner sep=1pt]
  \node[vertex] (v1) at (0.5, 1.5){};
  \node[vertex] (v2) at (1.5, 2.5){};
  \node[vertex] (v3) at (2.5, 0.5){};
  \node[vertex] (v4) at (3.5, 4.5){};
  \node[vertex] (v5) at (4.5, 5.5){};
  \node[vertex] (v6) at (5.5, 3.5){};
  \draw[gray, very thin] (0,0) grid (6,6);
  \draw (v1) (v2) (v3) (v4) (v5) (v6);
  \end{tikzpicture}}

\newcommand{\bcaocab}{
  \begin{tikzpicture}[baseline=0.6ex,scale=0.1]
  \tikzstyle{vertex}=[circle,fill=black, minimum size=2pt,inner sep=1pt]
  \node[vertex] (v1) at (0.5, 1.5){};
  \node[vertex] (v2) at (1.5, 2.5){};
  \node[vertex] (v3) at (2.5, 0.5){};
  \node[vertex] (v4) at (3.5, 5.5){};
  \node[vertex] (v5) at (4.5, 3.5){};
  \node[vertex] (v6) at (5.5, 4.5){};
  \draw[gray, very thin] (0,0) grid (6,6);
  \draw (v1) (v2) (v3) (v4) (v5) (v6);
  \end{tikzpicture}}


\newcommand{\baoabmab}{
  \begin{tikzpicture}[baseline=0.6ex,scale=0.1]
  \tikzstyle{vertex}=[circle,fill=black, minimum size=2pt,inner sep=1pt]
  \node[vertex] (v1) at (0.5, 1.5){};
  \node[vertex] (v2) at (1.5, 0.5){};
  \node[vertex] (v3) at (2.5, 4.5){};
  \node[vertex] (v4) at (3.5, 5.5){};
  \node[vertex] (v5) at (4.5, 2.5){};
  \node[vertex] (v6) at (5.5, 3.5){};
  \draw[gray, very thin] (0,0) grid (6,6);
  \draw (v1) (v2) (v3) (v4) (v5) (v6);
  \end{tikzpicture}}





%%%%%%%%%%%%%%%%%%%%%%%%%%%%%%%%%%%%%%%%%%%%%%%%%%%%%%%%%
%% GRID CLASSES
%%%%%%%%%%%%%%%%%%%%%%%%%%%%%%%%%%%%%%%%%%%%%%%%%%%%%%%%%

\newcommand{\cplusc}[2]{
  \begin{tikzpicture}[baseline=4ex, scale=0.8]
     % \filldraw[black] (0,2) circle (2pt);
      %\draw (-0.3,2.2) node {$\Z$};
      % \draw[-, very thick] (0.2,2.5) -- (0.2,0);
      % \draw[dashed] (0.2,2) -- (2.5,2);
      % \filldraw[black] (2.2,1.6) circle (2pt);
      % \draw (2.6,1.6) node {$\Z$};
      \draw (0,0) rectangle (1,1) node[pos=0.5]{\ensuremath{#1}};
      %\filldraw[black] (0.8,0.3) circle (2pt);
      \draw (1,1) rectangle (2,2) node[pos=0.5]{\ensuremath{#2}};
      % \filldraw[black] (0.8,0.3) circle (2pt);
      %\draw (1.2,0.2) node {$\Z$};
    \end{tikzpicture}}


 \newcommand{\cminusc}[2]{
  \begin{tikzpicture}[baseline=4ex, scale=0.8]
    %\filldraw[black] (0,2) circle (2pt);
      %\draw (-0.3,2.2) node {$\Z$};
      % \draw[-, very thick] (0.2,2.5) -- (0.2,0);
      % \draw[dashed] (0.2,2) -- (2.5,2);
      % \filldraw[black] (2.2,1.6) circle (2pt);
      % \draw (2.6,1.6) node {$\Z$};
      \draw (1,0) rectangle (2,1) node[pos=0.5]{\ensuremath{#1}};
      %\filldraw[black] (0.8,0.3) circle (2pt);
      \draw (0,1) rectangle (1,2) node[pos=0.5]{\ensuremath{#2}};
      % \filldraw[black] (0.8,0.3) circle (2pt);
      %\draw (1.2,0.2) node {$\Z$};
  \end{tikzpicture}}


